\documentclass[11pt,addpoints,answers]{exam}

% Packages
\usepackage{fullpage}
\usepackage{latexsym,amssymb,amsfonts,amsmath,mathrsfs,float, amsthm}
\usepackage[font=small,labelfont=bf, width=.618\textwidth]{caption} % custom captions
\usepackage{xcolor}
\usepackage{tikz, graphics, enumerate}
\usepackage{hyperref}


% Paragraph indentation and spacing
\newcommand{\parset}{
  \setlength{\parskip}{3mm}
    \setlength{\parindent}{0mm}}
  
% Probability
  \DeclareMathOperator{\var}{Var}
  \renewcommand{\Pr}{\mbox{\rm Pr}} 
  \newcommand{\Exp}{{\mathbb{E}}}
  \newcommand{\E}{\mathbb{E}}

% Sets 
  \newcommand{\R}{\mathbb{R}} % reals
  \newcommand{\C}{\mathbb{C}} % complex numbers
  \newcommand{\N}{\mathbb{N}} % natural numbers
  \newcommand{\Z}{\mathbb{Z}} % integers
  \newcommand{\F}{\mathbb{F}} % field
  \newcommand{\K}{\mathbb K} % field
  \newcommand{\T}{\mathbb T} % circle
\newcommand{\B}{\mathcal{B}} % ball
  \newcommand{\pmset}[1]{\{-1,1\}^{#1}} % hypercube in +-1 basis
  \newcommand{\bset}[1]{\{0,1\}^{#1}} % hypercube
  \newcommand{\sphere}[1]{S^{#1-1}} % real unit sphere of dimension #1
  \newcommand{\ball}[1]{B_{#1}} % real unit ball of dimension $1
  \newcommand{\Orth}[1]{O(\R^{#1})} % orthogonal group
  \DeclareMathOperator{\im}{im} % image
  \DeclareMathOperator{\vspan}{Span} % kernel
  \newcommand{\1}{\mathbf{1}}
  \DeclareMathOperator{\Cball}{\mathcal C}
  \DeclareMathOperator{\cay}{Cay} 
  \DeclareMathOperator{\sol}{Sol} 
\DeclareRobustCommand{\stirling}{\genfrac\{\}{0pt}{}}

% Miscellaneous
  \newcommand{\st}{:\,} % "such that" to define sets
  \newcommand{\ie}{{i.e.}}  
  \newcommand{\eg}{{e.g.}} 
  \newcommand{\eps}{\varepsilon}
  \newcommand{\ip}[1]{\langle #1 \rangle}
  \newcommand{\cF}{{\mathcal F}}   
  \DeclareMathOperator{\U}{\mathcal{U}}
  \DeclareMathOperator{\sign}{sign}
  \newcommand{\poly}{\mbox{\rm poly}}
  \newcommand{\ceil}[1]{\lceil{#1}\rceil}
  \newcommand{\floor}[1]{\lfloor{#1}\rfloor}
  \DeclareMathOperator{\Tr}{\mathsf{Tr}}
  \DeclareMathOperator{\diag}{diag}
  \DeclareMathOperator{\patt}{patt}
  \DeclareMathOperator{\argmin}{arg\,min}
  \DeclareMathOperator{\spec}{Spec}
  \DeclareMathOperator{\enc}{\sf Enc}
  \DeclareMathOperator{\dec}{\sf Dec}
  \DeclareMathOperator{\diam}{diam} 
  \DeclareMathOperator{\spn}{span} 
  \DeclareMathOperator{\geom}{gm}
\DeclareMathOperator{\sinc}{sinc}
\DeclareMathOperator{\disc}{disc}
\DeclareMathOperator*{\argmax}{arg\,max}
  \newcommand{\infnorm}{{\ell_\infty, \ldots, \ell_\infty}}
  \newcommand{\pnorm}{{\ell_p,\dots,\ell_p}}
  \newcommand{\tnorm}{{\ell_t,\dots,\ell_t}}
  
  
% Enviroments
  \newcommand{\beq}{\begin{equation}}
  \newcommand{\eeq}{\end{equation}}
  \newcommand{\beqn}{\begin{equation*}}
  \newcommand{\eeqn}{\end{equation*}}
  \newcommand{\beqr}{\begin{eqnarray}}
  \newcommand{\eeqr}{\end{eqnarray}}
  \newcommand{\beqrn}{\begin{eqnarray*}}
  \newcommand{\eeqrn}{\end{eqnarray*}}
  \newcommand{\bmline}{\begin{multline}}
  \newcommand{\emline}{\end{multline}}
  \newcommand{\bmlinen}{\begin{multline*}}
  \newcommand{\emlinen}{\end{multline*}}
  
  \makeatletter

% END OF SUPPLIED VARIABLES

  \chead{\large \textbf{Homework 4}}

  \lhead{\small
    \textbf{{Comp Sci 335}}}

  \rhead{\small \textbf{Name}: Caspar P., Rachel K., Melia T.}

  \setlength{\headheight}{20pt}
  \setlength{\headsep}{16pt}                       
  \headrule
% execute homework commands

\begin{document}

\pagestyle{head}                % put header on every page

\medskip
 
\section{$L_k$ is regular}

\noindent Theorem. $L_k$ is regular.

\medskip

\noindent Proof. We can show that a language is regular by writing a regular expression for it. The regular expression for $L_k$ is $(0 + 1)^* 1 (0 + 1)^{k-1}$. Therefore, $L_k$ is regular.

\section{Every DFA for $L_k$ has exponentially many states w.r.t $k$}

\noindent Theorem. Every DFA for $L_k$ has exponentially many states with respect to $k$.

\medskip

\noindent In other words, we will prove a lower bound on the number of states in any DFA $M$ (which has $|Q|$ states) which recognizes $L_k$: $|Q| \geq 2^k$.

\medskip

\noindent Proof.

\noindent Consider the set of all $k$-length strings $\{0, 1\}^k$. The size of the set is $2^k$. We will prove that given any $2$ strings $x$ and $y$ in this set, there exists a string $z$ that can be appended to $x$ and $y$ such that $xz \in L_k \wedge yz \notin L_k$ or vice versa. Therefore all $x$ and $y$ are mutually distinguishable.

\medskip

\noindent Consider some $x, y$ where the $i$th letter of $x$, which we denote as $x_i=1$, is different than $y_i=0$ ($x_i \neq y_i$). Then, there exists a string $z$ of length $i-1$ that may be appended to $x$ and $y$ to create $x_z$ and $y_z$ such that $x_z \in L_k \wedge y_z \notin L_k$, because $x_i=1$ and $y_i=0$ are $k$ positions from the end.

\medskip

\noindent Because all $2^k$ strings in this set are distinguishable from each other, then any DFA for the language $L_k$ must have exponentially many states with respect to $k$.

\end{document}
