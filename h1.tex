\documentclass[11pt,addpoints,answers]{exam}

% Packages
\usepackage{fullpage}
\usepackage{latexsym,amssymb,amsfonts,amsmath,mathrsfs,float, amsthm}
\usepackage[font=small,labelfont=bf, width=.618\textwidth]{caption} % custom captions
\usepackage{xcolor}
\usepackage{tikz, graphics, enumerate}
\usepackage{hyperref}


% Paragraph indentation and spacing
\newcommand{\parset}{
	\setlength{\parskip}{3mm}
  	\setlength{\parindent}{0mm}}
	
% Probability
  \DeclareMathOperator{\var}{Var}
  \renewcommand{\Pr}{\mbox{\rm Pr}}	
  \newcommand{\Exp}{{\mathbb{E}}}
  \newcommand{\E}{\mathbb{E}}

% Sets 
  \newcommand{\R}{\mathbb{R}} % reals
  \newcommand{\C}{\mathbb{C}} % complex numbers
  \newcommand{\N}{\mathbb{N}} % natural numbers
  \newcommand{\Z}{\mathbb{Z}} % integers
  \newcommand{\F}{\mathbb{F}} % field
  \newcommand{\K}{\mathbb K} % field
  \newcommand{\T}{\mathbb T} % circle
\newcommand{\B}{\mathcal{B}} % ball
  \newcommand{\pmset}[1]{\{-1,1\}^{#1}} % hypercube in +-1 basis
  \newcommand{\bset}[1]{\{0,1\}^{#1}} % hypercube
  \newcommand{\sphere}[1]{S^{#1-1}} % real unit sphere of dimension #1
  \newcommand{\ball}[1]{B_{#1}} % real unit ball of dimension $1
  \newcommand{\Orth}[1]{O(\R^{#1})} % orthogonal group
  \DeclareMathOperator{\im}{im} % image
  \DeclareMathOperator{\vspan}{Span} % kernel
  \newcommand{\1}{\mathbf{1}}
  \DeclareMathOperator{\Cball}{\mathcal C}
  \DeclareMathOperator{\cay}{Cay} 
  \DeclareMathOperator{\sol}{Sol} 
\DeclareRobustCommand{\stirling}{\genfrac\{\}{0pt}{}}

% Miscellaneous
  \newcommand{\st}{:\,} % "such that" to define sets
  \newcommand{\ie}{{i.e.}}  
  \newcommand{\eg}{{e.g.}} 
  \newcommand{\eps}{\varepsilon}
  \newcommand{\ip}[1]{\langle #1 \rangle}
  \newcommand{\cF}{{\mathcal F}}   
  \DeclareMathOperator{\U}{\mathcal{U}}
  \DeclareMathOperator{\sign}{sign}
  \newcommand{\poly}{\mbox{\rm poly}}
  \newcommand{\ceil}[1]{\lceil{#1}\rceil}
  \newcommand{\floor}[1]{\lfloor{#1}\rfloor}
  \DeclareMathOperator{\Tr}{\mathsf{Tr}}
  \DeclareMathOperator{\diag}{diag}
  \DeclareMathOperator{\patt}{patt}
  \DeclareMathOperator{\argmin}{arg\,min}
  \DeclareMathOperator{\spec}{Spec}
  \DeclareMathOperator{\enc}{\sf Enc}
  \DeclareMathOperator{\dec}{\sf Dec}
  \DeclareMathOperator{\diam}{diam} 
  \DeclareMathOperator{\spn}{span} 
  \DeclareMathOperator{\geom}{gm}
\DeclareMathOperator{\sinc}{sinc}
\DeclareMathOperator{\disc}{disc}
\DeclareMathOperator*{\argmax}{arg\,max}
  \newcommand{\infnorm}{{\ell_\infty, \ldots, \ell_\infty}}
  \newcommand{\pnorm}{{\ell_p,\dots,\ell_p}}
  \newcommand{\tnorm}{{\ell_t,\dots,\ell_t}}
  
  
% Enviroments
  \newcommand{\beq}{\begin{equation}}
  \newcommand{\eeq}{\end{equation}}
  \newcommand{\beqn}{\begin{equation*}}
  \newcommand{\eeqn}{\end{equation*}}
  \newcommand{\beqr}{\begin{eqnarray}}
  \newcommand{\eeqr}{\end{eqnarray}}
  \newcommand{\beqrn}{\begin{eqnarray*}}
  \newcommand{\eeqrn}{\end{eqnarray*}}
  \newcommand{\bmline}{\begin{multline}}
  \newcommand{\emline}{\end{multline}}
  \newcommand{\bmlinen}{\begin{multline*}}
  \newcommand{\emlinen}{\end{multline*}}
  
  \makeatletter

% END OF SUPPLIED VARIABLES

  \chead{\large \textbf{Homework 1}}

  \lhead{\small
    \textbf{{Comp Sci 335}}}

  \rhead{\small \textbf{Name}: Caspar P., Rachel K., Melia T.}

  \setlength{\headheight}{20pt}
  \setlength{\headsep}{16pt}                       
  \headrule
% execute homework commands

\begin{document}

\pagestyle{head}                % put header on every page

\medskip
 
\section{Goal}

\noindent Theorem. $\exists \ h : P(S) \rightarrow G$, or a bijection $h$ from the powerset of $S$ to $G$. $h$ is used to prove that $|S| <|G|$.  

\noindent Proof. 

\section{Definition of h}
Let $h : P(S) \rightarrow G$ be defined as:

\begin{flalign}
    \forall p \in P(S) \ . \ h(p) = g \ \text{s.t.}
    \\ \forall s_1 \in p \ . \ g(s_1) = 1
    \\ \forall s_2 \in (S - p) \ . \  g(s_2) = -1
\end{flalign}

\noindent Or, $h(p)$ is a function $g : S \rightarrow \{-1, 1\}$ that returns $1$ for all $s_1$ in $p$, and $-1$ otherwise.

\section{h is injective}
Let $a, b \in P(S)$ such that $h(a) = h(b) = g$. By the definition of $h$, $g : S \rightarrow \{-1, 1\}$ is a function that map elements of the same $p \subseteq S$ to $1$, and all other elements $s \in S$ to $-1$.

\begin{flalign}
    \forall s_1 \in p \ . \ g(s_1) = 1
    \\ \forall s_2 \in (S - p) \ . \  g(s_2) = -1
\end{flalign}

\noindent By construction, there are no $a, b \in P(S)$ such that $a \neq b$ if $h(a) = h(b)$.

\noindent Since $h(a) = h(b) \implies a = b$, $h$ is injective.

\section{h is surjective}

Every function $g \in G$ can be described by the set $p$, where 

\begin{flalign}
    \forall g \in G \ . \  \ \exists \ p \in P(S) \ \text{s.t.}
    \\ \forall s_1 \in p \ . \  \ g(s_1) = 1
    \\ \forall s_2 \in (S  - p) \ . \ g(s_2) = -1
\end{flalign}

\noindent Since $\forall p \ . \ p \subseteq S \implies p \in P(S)$, $\forall g \in G \ . \ \exists \ p \in P(S)$ such that $h(p) = g$.

\section{Cardinalities of S and G}

Since $h$ is both injective and surjective, $h$ is a bijection. Therefore, $|P(S)| = |G|$. By the lemma $|S| < |P(S)|$, $|S| < |G|$. $\qed$

\end{document}