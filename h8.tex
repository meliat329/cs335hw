\documentclass[11pt,addpoints,answers]{exam}

% Packages
\usepackage{fullpage}
\usepackage{latexsym,amssymb,amsfonts,amsmath,mathrsfs,float, amsthm}
\usepackage[font=small,labelfont=bf, width=.618\textwidth]{caption} % custom captions
\usepackage{xcolor}
\usepackage{tikz, graphics, enumerate}
\usepackage{hyperref}


% Paragraph indentation and spacing
\newcommand{\parset}{
	\setlength{\parskip}{3mm}
  	\setlength{\parindent}{0mm}}
	
% Probability
  \DeclareMathOperator{\var}{Var}
  \renewcommand{\Pr}{\mbox{\rm Pr}}	
  \newcommand{\Exp}{{\mathbb{E}}}
  \newcommand{\E}{\mathbb{E}}

% Sets 
  \newcommand{\R}{\mathbb{R}} % reals
  \newcommand{\C}{\mathbb{C}} % complex numbers
  \newcommand{\N}{\mathbb{N}} % natural numbers
  \newcommand{\Z}{\mathbb{Z}} % integers
  \newcommand{\F}{\mathbb{F}} % field
  \newcommand{\K}{\mathbb K} % field
  \newcommand{\T}{\mathbb T} % circle
\newcommand{\B}{\mathcal{B}} % ball
  \newcommand{\pmset}[1]{\{-1,1\}^{#1}} % hypercube in +-1 basis
  \newcommand{\bset}[1]{\{0,1\}^{#1}} % hypercube
  \newcommand{\sphere}[1]{S^{#1-1}} % real unit sphere of dimension #1
  \newcommand{\ball}[1]{B_{#1}} % real unit ball of dimension $1
  \newcommand{\Orth}[1]{O(\R^{#1})} % orthogonal group
  \DeclareMathOperator{\im}{im} % image
  \DeclareMathOperator{\vspan}{Span} % kernel
  \newcommand{\1}{\mathbf{1}}
  \DeclareMathOperator{\Cball}{\mathcal C}
  \DeclareMathOperator{\cay}{Cay} 
  \DeclareMathOperator{\sol}{Sol} 
\DeclareRobustCommand{\stirling}{\genfrac\{\}{0pt}{}}

% Miscellaneous
  \newcommand{\st}{:\,} % "such that" to define sets
  \newcommand{\ie}{{i.e.}}  
  \newcommand{\eg}{{e.g.}} 
  \newcommand{\eps}{\varepsilon}
  \newcommand{\ip}[1]{\langle #1 \rangle}
  \newcommand{\cF}{{\mathcal F}}   
  \DeclareMathOperator{\U}{\mathcal{U}}
  \DeclareMathOperator{\sign}{sign}
  \newcommand{\poly}{\mbox{\rm poly}}
  \newcommand{\ceil}[1]{\lceil{#1}\rceil}
  \newcommand{\floor}[1]{\lfloor{#1}\rfloor}
  \DeclareMathOperator{\Tr}{\mathsf{Tr}}
  \DeclareMathOperator{\diag}{diag}
  \DeclareMathOperator{\patt}{patt}
  \DeclareMathOperator{\argmin}{arg\,min}
  \DeclareMathOperator{\spec}{Spec}
  \DeclareMathOperator{\enc}{\sf Enc}
  \DeclareMathOperator{\dec}{\sf Dec}
  \DeclareMathOperator{\diam}{diam} 
  \DeclareMathOperator{\spn}{span} 
  \DeclareMathOperator{\geom}{gm}
\DeclareMathOperator{\sinc}{sinc}
\DeclareMathOperator{\disc}{disc}
\DeclareMathOperator*{\argmax}{arg\,max}
  \newcommand{\infnorm}{{\ell_\infty, \ldots, \ell_\infty}}
  \newcommand{\pnorm}{{\ell_p,\dots,\ell_p}}
  \newcommand{\tnorm}{{\ell_t,\dots,\ell_t}}
  
  
% Enviroments
  \newcommand{\beq}{\begin{equation}}
  \newcommand{\eeq}{\end{equation}}
  \newcommand{\beqn}{\begin{equation*}}
  \newcommand{\eeqn}{\end{equation*}}
  \newcommand{\beqr}{\begin{eqnarray}}
  \newcommand{\eeqr}{\end{eqnarray}}
  \newcommand{\beqrn}{\begin{eqnarray*}}
  \newcommand{\eeqrn}{\end{eqnarray*}}
  \newcommand{\bmline}{\begin{multline}}
  \newcommand{\emline}{\end{multline}}
  \newcommand{\bmlinen}{\begin{multline*}}
  \newcommand{\emlinen}{\end{multline*}}
  
  \makeatletter

% END OF SUPPLIED VARIABLES

  \chead{\large \textbf{Homework 8}}

  \lhead{\small
    \textbf{{Comp Sci 335}}}

  \rhead{\small \textbf{Name}: Caspar P., Rachel K.}

  \setlength{\headheight}{20pt}
  \setlength{\headsep}{16pt}                       
  \headrule
% execute homework commands

\begin{document}

\pagestyle{head}                % put header on every page

\medskip
 
\section{Claim}

We will show that $C(xy) \leq 2 C(x) + C(y) + c$ by showing that there exists a TM $M^\prime$ which can write $xy$ to its tape, by taking in a special concatenation of $<x, y>$ and extracting $x$ and $y$ from it.

\section{Concatenation algorithm}

The concatenation algorithm takes in an ordered pair of strings $<x,y>$ and produces a new string $0^{\text{len}(x)}1xy$. From this string of length $2x + y + 1$, it is possible to recover $x$ and $y$.

\section{$M^\prime$ to extract $xy$ from the concatenation}

Let $M^\prime$ be a Turning machine that takes in the concatenation $0^{\text{len}(x)}1xy$ and returns $xy$.

\begin{enumerate}
    \item Read in all the 0s in the input until a 1 is reached
    \item Repeat until there are no more unmarked 0s before the 1: mark the first unmarked character after the 1, then return to the start of the string and mark the first unmarked character before the 1.
    \item Once all the 0s are marked off, the marked characters after the 1 are x, and the unmarked characters after the 1 are y.
\end{enumerate}

\section{Kolmogorov complexity of $M^\prime$}

From the definition of Kolmogorov complexity, there exists a TM $M_1$ such that $M_1(a) = x$, and analogously there exists a TM $M_2$ such that $M_2(b) = y$. Then:
\begin{align*}
    C(x) = \text{len}(a) \\
    C(y) = \text{len}(b)
\end{align*}

From our definition above,

\begin{align*}
    C_{M^\prime} \leq 2 \text{len}(a) + \text{len}(b) + c_{M^\prime}
\end{align*}

By substitution,

\begin{align*}
     C_{M^\prime}(xy) \leq 2 C(x) + C(y) + c_{M^\prime}
\end{align*}

By the invariance theorem, 

\begin{align*}
    C(xy) \leq 2 C(x) + C(y) + c_{M^\prime}
\end{align*}

\end{document} 