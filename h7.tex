\documentclass[11pt,addpoints,answers]{exam}

% Packages
\usepackage{fullpage}
\usepackage{latexsym,amssymb,amsfonts,amsmath,mathrsfs,float, amsthm}
\usepackage[font=small,labelfont=bf, width=.618\textwidth]{caption} % custom captions
\usepackage{xcolor}
\usepackage{tikz, graphics, enumerate}
\usepackage{hyperref}


% Paragraph indentation and spacing
\newcommand{\parset}{
  \setlength{\parskip}{3mm}
    \setlength{\parindent}{0mm}}
  
% Probability
  \DeclareMathOperator{\var}{Var}
  \renewcommand{\Pr}{\mbox{\rm Pr}} 
  \newcommand{\Exp}{{\mathbb{E}}}
  \newcommand{\E}{\mathbb{E}}

% Sets 
  \newcommand{\R}{\mathbb{R}} % reals
  \newcommand{\C}{\mathbb{C}} % complex numbers
  \newcommand{\N}{\mathbb{N}} % natural numbers
  \newcommand{\Z}{\mathbb{Z}} % integers
  \newcommand{\F}{\mathbb{F}} % field
  \newcommand{\K}{\mathbb K} % field
  \newcommand{\T}{\mathbb T} % circle
\newcommand{\B}{\mathcal{B}} % ball
  \newcommand{\pmset}[1]{\{-1,1\}^{#1}} % hypercube in +-1 basis
  \newcommand{\bset}[1]{\{0,1\}^{#1}} % hypercube
  \newcommand{\sphere}[1]{S^{#1-1}} % real unit sphere of dimension #1
  \newcommand{\ball}[1]{B_{#1}} % real unit ball of dimension $1
  \newcommand{\Orth}[1]{O(\R^{#1})} % orthogonal group
  \DeclareMathOperator{\im}{im} % image
  \DeclareMathOperator{\vspan}{Span} % kernel
  \newcommand{\1}{\mathbf{1}}
  \DeclareMathOperator{\Cball}{\mathcal C}
  \DeclareMathOperator{\cay}{Cay} 
  \DeclareMathOperator{\sol}{Sol} 
\DeclareRobustCommand{\stirling}{\genfrac\{\}{0pt}{}}

% Miscellaneous
  \newcommand{\st}{:\,} % "such that" to define sets
  \newcommand{\ie}{{i.e.}}  
  \newcommand{\eg}{{e.g.}} 
  \newcommand{\eps}{\varepsilon}
  \newcommand{\ip}[1]{\langle #1 \rangle}
  \newcommand{\cF}{{\mathcal F}}   
  \DeclareMathOperator{\U}{\mathcal{U}}
  \DeclareMathOperator{\sign}{sign}
  \newcommand{\poly}{\mbox{\rm poly}}
  \newcommand{\ceil}[1]{\lceil{#1}\rceil}
  \newcommand{\floor}[1]{\lfloor{#1}\rfloor}
  \DeclareMathOperator{\Tr}{\mathsf{Tr}}
  \DeclareMathOperator{\diag}{diag}
  \DeclareMathOperator{\patt}{patt}
  \DeclareMathOperator{\argmin}{arg\,min}
  \DeclareMathOperator{\spec}{Spec}
  \DeclareMathOperator{\enc}{\sf Enc}
  \DeclareMathOperator{\dec}{\sf Dec}
  \DeclareMathOperator{\diam}{diam} 
  \DeclareMathOperator{\spn}{span} 
  \DeclareMathOperator{\geom}{gm}
\DeclareMathOperator{\sinc}{sinc}
\DeclareMathOperator{\disc}{disc}
\DeclareMathOperator*{\argmax}{arg\,max}
  \newcommand{\infnorm}{{\ell_\infty, \ldots, \ell_\infty}}
  \newcommand{\pnorm}{{\ell_p,\dots,\ell_p}}
  \newcommand{\tnorm}{{\ell_t,\dots,\ell_t}}
  
  
% Enviroments
  \newcommand{\beq}{\begin{equation}}
  \newcommand{\eeq}{\end{equation}}
  \newcommand{\beqn}{\begin{equation*}}
  \newcommand{\eeqn}{\end{equation*}}
  \newcommand{\beqr}{\begin{eqnarray}}
  \newcommand{\eeqr}{\end{eqnarray}}
  \newcommand{\beqrn}{\begin{eqnarray*}}
  \newcommand{\eeqrn}{\end{eqnarray*}}
  \newcommand{\bmline}{\begin{multline}}
  \newcommand{\emline}{\end{multline}}
  \newcommand{\bmlinen}{\begin{multline*}}
  \newcommand{\emlinen}{\end{multline*}}
  
  \makeatletter

% END OF SUPPLIED VARIABLES

  \chead{\large \textbf{Homework 7}}

  \lhead{\small
    \textbf{{Comp Sci 335}}}

  \rhead{\small \textbf{Name}: Caspar P., Rachel K.}

  \setlength{\headheight}{20pt}
  \setlength{\headsep}{16pt}                       
  \headrule
% execute homework commands

\begin{document}

\pagestyle{head}                % put header on every page

\medskip
 
\section{Claim}

We will show that the $101011$ problem is not co-recognizable by showing that HALT$_{\text{TM}}$ is mapping reducible to the language L, where:

\begin{align*}
    \text{HALT}_{\text{TM}} = \{ \langle M, w \rangle \ | \ M \text{ halts on some input } w \} \\
    \text{L}_{\text{TM}} = \{ \langle M \rangle \ | \ M \text{ is a Turing machine that accepts the string } 101011 \}
\end{align*}

\noindent By showing $\text{HALT}_\text{TM} \leq_m L_\text{TM}$, and because HALT is not co-recognizable, then we show that $L_\text{TM}$ is not co-recognizable.

\section{Definition of $f$}

$f$: Given any machine $M$ and input $w$, we construct the machine $M^\prime$ as follows:

\[
M^\prime (x) =
\left\{
\begin{array}{ll}
    x = w \wedge M(w) \text{ halts} & M^\prime \text{ accepts} \\
    x \neq w \wedge M(w) \text{ halts} & M^\prime \text{ rejects} \\
    M(w) \text{ does not halt} & M^\prime \text{ does not halt}
\end{array}
\right.
\]

\noindent First, $M^\prime$ runs $M$ on $w$.

\noindent If $M(w)$ halts, then $M^\prime$ will compare $x$ to $101011$: accept if $x == 101011$ and reject otherwise.

\noindent If $M(w)$ does not halt, then $M^\prime$ does not halt.

\section{Proof that $f$ is a mapping reduction}

To show that $f$ is a valid mapping reduction, we must show that

\begin{align*}
    1. \ (M, w) \in \text{HALT}_\text{TM} \implies M^\prime \in \text{L}_\text{TM} \\
    2. \ (M, w) \notin \text{HALT}_\text{TM} \implies M^\prime \notin \text{L}_\text{TM}
\end{align*}

\noindent Proof.

\begin{enumerate}
    \item If $(M, w) \in \text{HALT}_\text{TM}$, then $M(w)$ halts. Then, $M^\prime$ is a decider for L$_\text{TM}$ by comparing $x$ to $101011$, so $M^\prime \in \text{L}_\text{TM}$.
    \item If $(M, w) \notin \text{HALT}_\text{TM}$, then $M(w)$ does not halt and $M^\prime$ must run forever. Since $M^\prime$ will not accept any input including $101011$, then $M^\prime$ does not decide $L$, and $M^\prime \notin \text{L}_\text{TM}$.
\end{enumerate}

\end{document} 
