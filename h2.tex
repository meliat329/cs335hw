\documentclass[11pt,addpoints,answers]{exam}

% Packages
\usepackage{fullpage}
\usepackage{latexsym,amssymb,amsfonts,amsmath,mathrsfs,float, amsthm}
\usepackage[font=small,labelfont=bf, width=.618\textwidth]{caption} % custom captions
\usepackage{xcolor}
\usepackage{tikz, graphics, enumerate}
\usepackage{hyperref}


% Paragraph indentation and spacing
\newcommand{\parset}{
	\setlength{\parskip}{3mm}
  	\setlength{\parindent}{0mm}}
	
% Probability
  \DeclareMathOperator{\var}{Var}
  \renewcommand{\Pr}{\mbox{\rm Pr}}	
  \newcommand{\Exp}{{\mathbb{E}}}
  \newcommand{\E}{\mathbb{E}}

% Sets 
  \newcommand{\R}{\mathbb{R}} % reals
  \newcommand{\C}{\mathbb{C}} % complex numbers
  \newcommand{\N}{\mathbb{N}} % natural numbers
  \newcommand{\Z}{\mathbb{Z}} % integers
  \newcommand{\F}{\mathbb{F}} % field
  \newcommand{\K}{\mathbb K} % field
  \newcommand{\T}{\mathbb T} % circle
\newcommand{\B}{\mathcal{B}} % ball
  \newcommand{\pmset}[1]{\{-1,1\}^{#1}} % hypercube in +-1 basis
  \newcommand{\bset}[1]{\{0,1\}^{#1}} % hypercube
  \newcommand{\sphere}[1]{S^{#1-1}} % real unit sphere of dimension #1
  \newcommand{\ball}[1]{B_{#1}} % real unit ball of dimension $1
  \newcommand{\Orth}[1]{O(\R^{#1})} % orthogonal group
  \DeclareMathOperator{\im}{im} % image
  \DeclareMathOperator{\vspan}{Span} % kernel
  \newcommand{\1}{\mathbf{1}}
  \DeclareMathOperator{\Cball}{\mathcal C}
  \DeclareMathOperator{\cay}{Cay} 
  \DeclareMathOperator{\sol}{Sol} 
\DeclareRobustCommand{\stirling}{\genfrac\{\}{0pt}{}}

% Miscellaneous
  \newcommand{\st}{:\,} % "such that" to define sets
  \newcommand{\ie}{{i.e.}}  
  \newcommand{\eg}{{e.g.}} 
  \newcommand{\eps}{\varepsilon}
  \newcommand{\ip}[1]{\langle #1 \rangle}
  \newcommand{\cF}{{\mathcal F}}   
  \DeclareMathOperator{\U}{\mathcal{U}}
  \DeclareMathOperator{\sign}{sign}
  \newcommand{\poly}{\mbox{\rm poly}}
  \newcommand{\ceil}[1]{\lceil{#1}\rceil}
  \newcommand{\floor}[1]{\lfloor{#1}\rfloor}
  \DeclareMathOperator{\Tr}{\mathsf{Tr}}
  \DeclareMathOperator{\diag}{diag}
  \DeclareMathOperator{\patt}{patt}
  \DeclareMathOperator{\argmin}{arg\,min}
  \DeclareMathOperator{\spec}{Spec}
  \DeclareMathOperator{\enc}{\sf Enc}
  \DeclareMathOperator{\dec}{\sf Dec}
  \DeclareMathOperator{\diam}{diam} 
  \DeclareMathOperator{\spn}{span} 
  \DeclareMathOperator{\geom}{gm}
\DeclareMathOperator{\sinc}{sinc}
\DeclareMathOperator{\disc}{disc}
\DeclareMathOperator*{\argmax}{arg\,max}
  \newcommand{\infnorm}{{\ell_\infty, \ldots, \ell_\infty}}
  \newcommand{\pnorm}{{\ell_p,\dots,\ell_p}}
  \newcommand{\tnorm}{{\ell_t,\dots,\ell_t}}
  
  
% Enviroments
  \newcommand{\beq}{\begin{equation}}
  \newcommand{\eeq}{\end{equation}}
  \newcommand{\beqn}{\begin{equation*}}
  \newcommand{\eeqn}{\end{equation*}}
  \newcommand{\beqr}{\begin{eqnarray}}
  \newcommand{\eeqr}{\end{eqnarray}}
  \newcommand{\beqrn}{\begin{eqnarray*}}
  \newcommand{\eeqrn}{\end{eqnarray*}}
  \newcommand{\bmline}{\begin{multline}}
  \newcommand{\emline}{\end{multline}}
  \newcommand{\bmlinen}{\begin{multline*}}
  \newcommand{\emlinen}{\end{multline*}}
  
  \makeatletter

% END OF SUPPLIED VARIABLES
  \lhead{\small
    \textbf{{Comp Sci 335: Homework 2}}}

  \rhead{\small \textbf{Names}: Caspar Popova, Rachel Kantor, Melia Tomlinson}

  \setlength{\headheight}{20pt}
  \setlength{\headsep}{16pt}                       
  \headrule
% execute homework commands

\begin{document}

\pagestyle{head}                % put header on every page

\medskip
 
\section{Prove S is regular}

\noindent Lemma. For any finite string $s$, $\exists$ a DFA that recognizes exactly the string 's'.

\noindent Corallary. $\forall s_j \in S, \exists \ M_j$ s.t. $L(M_j) = \{s_j\}$. In other words, for all strings s in S, there exists a DFA M that recognizes the language containing the set $\{s\}$.

\noindent Then $\forall s_i \in S . \exists \ L_j$ s.t. $L$ is regular.

\noindent By the union closure property, $ L_S = \cup L_j$ is regular.

\noindent Therefore there is a regular languge $L_S$ which recognizes $S$, i.e. $L_S(M) = S$.

\section{Prove L is regular}

\noindent Theorem. The language L is regular, where $L = \{s_1 \cdots s_n | n \geq \N$ and $n \geq 1$ and each $s_j \in S$ is accepted by $L_S \}$.

\noindent In other worlds, L can be recognized by a DFA.

\noindent We will show 1) $L(N) \subseteq L$ and 2) $L(N) \supseteq L$, by which $L(N)=L$.

\section{Definition of N}

We will define an NFA $N$ which recognizes $L$ by starting with the DFA $M$ that recognizes $L_S$.

\noindent $M = \{Q, \Sigma, \delta, q_1, F \}$

\noindent $N = \{Q^\prime, \Sigma, \delta^\prime, q_0, F \}$

\noindent Where:

\noindent $Q^\prime = Q \cup \{q_0\}$

$$
\delta^\prime(q, a) =
\begin{cases}
	\{ \delta(q, a) \}		\quad q \in Q \wedge a \neq \epsilon \\
	\{ q_1 \}			\quad q \in F \wedge a = \epsilon \\
	\{ q_1 \}			\quad q = q_0 \wedge a = \epsilon \\
	\emptyset			\quad \text{else}
\end{cases}
$$

\section{Part one}

\noindent Assume $w = s_1, \cdots, s_k$ s.t. $w \in L$ where $s_1 \cdots s_k \in S$.

\noindent We show $N$ accepts $w$ by induction on $k$.

\medskip

\noindent Base Case

\noindent $k = 1 \implies w = s_1$ because $s_j \in S \implies$ because $\delta(s_0, \epsilon) = \{ q_1 \}$ there exists a path in the original DFA from $q_1$ to some final state, then N accepts w. // todo

\medskip

\noindent Inductive Step

\noindent Let $v = v_1 \cdots v_k$ s.t. $\exists \ v_1 \cdots v_j \in L_S$ and $N$ accepts $v$.

\noindent Let $u = u_1 \cdots u_k u_{k + 1} \in L$ and $u_j \in L_S$.

\noindent Proof.

\noindent Since $N$ accepts $u_1 \cdots u_k$ and $M$ accepts $u_{k + 1}$ then $N$ accepts $u_{k+1}$ by construction.

\section{Part two}

\noindent Assume $\exists \ w$ s.t. $N$ accepts $w$. We want to prove that $w \in L$.

\noindent I.H. $N$ accepts $u$ and takes at most $k \ \epsilon$-transitions and $1 \leq k \leq n$, then $w \in L$.

\noindent By induction on $k$.

\medskip

\noindent Base Case

\noindent $k = 1$ then $u \in L_S$ ($M$ accepts $u$) then $u \in L$.

\medskip

\noindent Inductive step.

\noindent Let $w$ s.t. $w$ is accepted by $N$ with $k \ \epsilon$-transitions,

\noindent Let $w = uv$ where $v$ is the substring read after the last $\epsilon$-transition.

\noindent $u \in L(N)$ by construction by IH, so $u \in L$.

\noindent $v \in L_s$, so by the definition of $u$, the computation of $q_1$ is contained inside $M$.

\noindent Therefore, $w = uv \in L$.

\end{document}